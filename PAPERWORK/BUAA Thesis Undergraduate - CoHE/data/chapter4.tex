% !Mode:: "TeX:UTF-8"
\chapter{实验设计}

\section{先导实验}

\subsection{实验1:眼动确认信号筛选}
\label{pilot-study-1}

此先导实验的目的是从最为常见且自然的两种主动眼动信号(单眼眨眼和快速两次眨眼)中确定一个最为高效并且带来最小使用压力的眼动确认信号。

我们招募了6名参试者;为了保证公平性和均一性,我们的此次实验组由3名男性和3名女性构成。参试者在开始实验前填写第一个个人信息调查问卷,以询问性别、工作机构、年龄、VR使用经验和能否自然地完成单眼眨眼的动作。

每名参试者将使用不同的两个眼动确认信号分两次完成同样的一个任务。该任务内容是,在虚拟环境中会依次在随机时刻和随机位置出现20个小球,参试者需要使用当次实验规定的眼动信号(单眼眨眼或快速两次眨眼)选择以消灭它们。为了保证每次实验结果的可比较性,我们规定:每个小球只有在参与者瞄准并且发出确认信号后才能消灭;下一个小球只有在当前小球被消灭后才生成;在每次发出确认信号前,参与者必须按下一次键盘上的空格键,以记录实际的信号尝试次数。

在实验完成后,每名参试者将分别针对两种确认信号完成一个NASA-TLX任务负担表。我们将针对以下三个客观度量评估实验结果:

\begin{itemize}[wide]
	\item 任务完成时间(毫秒);
	\item 任务负担(NASA-TLX分数);
	\item 信号反馈精确系数(Feedback Accuracy Index, FAI)。我们规定在一次实验中,参试者实际消灭的小球数量为 $N$ ,总共发出的信号尝试次数为 $M$;则有FAI的计算规则:
	\begin{equation}
	FAI = \frac{N}{M}
	\label{formula-4-1}
	\end{equation}
\end{itemize}

本次先导实验的实验结果汇报见表\ref{table-4-1}。

我们采用曼-惠特尼U检验(Mann-Whitney U Test)来分析数据的显著差异性;该检验假设两个样本分别来自除了总体均值意外完全相同的两个总体,目的是检验这两个总体的均值是否有显著的差异。对于任务完成时间,检验结果是 $U = 18,\ p < 0.05$ ;对于任务负担,检验结果是 $U = 42,\ p < 0.05$ ;对于FAI,检验结果是 $U = 15.5,\ p < 0.05$ 。可以发现,三组结果皆存在显著差异;而快速两次眨眼均值表现更为理想,故可以确定快速两次眨眼显著优于单眼眨眼。

\begin{table}[h!]
\centering
\begin{tabular}{ccccccc}
\multirow{2}{*}{\textbf{参试者}} & \multicolumn{3}{c}{\textbf{单眼眨眼}} & \multicolumn{3}{c}{\textbf{快速两次眨眼}} \\
  & 完成时间(ms) & NASA-TLX & FAI  & 完成时间(ms) & NASA-TLX & FAI  \\
1 & 187621   & 92       & 0.43 & 105832   & 71       & 0.79 \\
2 & 80122    & 71       & 0.87 & 113475   & 68       & 0.88 \\
3 & 161298   & 83       & 0.39 & 92103    & 79       & 0.91 \\
4 & 89172    & 62       & 0.91 & 125764   & 73       & 0.76 \\
5 & 90817    & 77       & 0.81 & 109675   & 91       & 0.86 \\
6 & 80012    & 89       & 0.49 & 98122    & 80       & 0.83
\end{tabular}
\caption{先导实验1:眼动确认信号筛选 - 实验结果}
\label{table-4-1}
\end{table}

\subsection{实验2:眼动操纵视线驻留检测优化}

此先导实验的目的是是否有必要引入眼动操纵注视停留检测的优化。

此次实验的实验组和先导实验1(详见\ref{pilot-study-1})一致。每名参试者将分两次完成同样的一个任务;在两次的实验中,我们会随机在某一次引入优化以消除主观的心理作用干扰。该任务内容和先导实验1一致;我们规定在此次先导实验中,两次任务都使用快速两次眨眼作为确认信号。

\begin{table}[b!]
\centering
\begin{tabular}{cccc}
\multirow{2}{*}{\textbf{参试者}} & \multirow{2}{*}{\textbf{\begin{tabular}[c]{@{}c@{}}引入优化组别\\ (参试者不可见)\end{tabular}}} & \textbf{含优化} & \textbf{不含优化} \\
  &   & NASA-TLX & NASA-TLX \\
1 & 1 & 73       & 80       \\
2 & 2 & 47       & 53       \\
3 & 1 & 68       & 62       \\
4 & 2 & 75       & 83       \\
5 & 2 & 70       & 79       \\
6 & 1 & 82       & 86      
\end{tabular}
\caption{先导实验2:眼动操纵视线驻留检测优化 - 实验结果}
\label{table-4-2}
\end{table}

在实验完成后,每名参试者将分别针对两次任务完成一个NASA-TLX任务负担表;由于用户在实验过程中并不知情某次任务是否引入优化,所以我们可以认为结果是客观的。我们将针对任务负担评估实验结果。

本次先导实验的实验结果汇报见表\ref{table-4-2}。

我们同样采用曼-惠特尼U检验来分析数据的显著差异性。对于任务负担,检验结果是 $U = 32,\ p < 0.05$ 。可以发现,两组结果存在显著差异,且引入优化组别和NASA-TLX结果不存在明显的相关性;而引入优化的任务负担均值表现更为理想,故此可以确定引入优化是必要的。

\section{用户实验}

\subsection{实验1:单物体位移对接实验}

\subsection{实验2:单物体操纵对接实验}
