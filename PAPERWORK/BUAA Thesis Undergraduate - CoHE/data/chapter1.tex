% !Mode:: "TeX:UTF-8"
\chapter{绪论}

\section{课题来源}

课题来源于北京航空航天大学虚拟现实技术与系统国家重点实验室。

\section{研究背景}

虚拟现实(Virtual Reality)技术,简称 VR,是一种利用计算机技术来模拟生活环境或创造虚拟现实的新型多媒体技术,是扩展现实(Extended Reality)技术的一个分支。目前主流的 VR 设备可利用头戴式显示器建立起一个完全虚拟的三维空间。使用者在这个虚拟的环境里进行交互操作时,计算机可以立即进行高度实时的、复杂的运算,将精确的三维影像传回,让使用者身处完全的沉浸式视觉环境中。该技术整合了计算机图形学、仿真模拟、人工智能以及并行计算等技术的最新发展成果,是一种融合多种先进技术的模拟系统。然而学界和工业界依旧留存着许多非常关键的VR相关问题亟待解决,拥有非常大的研究价值。

就目前而言,限制 VR 普及和发展的较为直接的阻碍,除开较高的市场售价,则来自于其仍旧较低的易用性,即虚拟环境中对物体的操控和交互的准确度依旧不容乐观,或是操作指令和交互动作过于复杂繁琐。这个缺陷直接降低了用户对 VR 技术的接受度和使用期望。因此,VR 中的对象操纵方法的优越性是提高其使用体验和普及度的基本问题之一。

许多研究者已经对虚拟现实的对象操纵方法进行了大量的研究,但仍有较大的提升空间。常规的对象操纵动作包括点击、按压、抓取、移动、释放等,其对应对象的直接具体表现主要为创建、销毁、位移、形变、旋转和缩放。对象操纵的速度、准确性、学习成本、使用压力和多样性将直接影响应用程序的效果,而在虚拟环境中实现高效且易用的对象操纵具有一定的挑战性。因此,本研究拟针对虚拟现实中对象操纵的关键问题进行研究,旨在提出相较于目前国际一流水准方法更加高效易用的基于头眼协同的虚拟现实对象操纵方法。

\section{研究目标}

本研究的主要探究问题是如何确定一套快速、准确、易用的虚拟现实头眼协同对象操纵方法。这个问题的主要难点包含:(1)通过眼动可获取的信号有限;(2)眼动信号不稳定,并且由于本能动作(如眨眼)干扰,眼动信号解析难度大;(3)眼动操纵使用负担非常大,现有方法操作流程复杂并且需要视线高度集中,容易产生眼球生理性疲劳。

基于此考虑,我们需要以尽可能少的头眼动作组合和尽可能小的专注度要求完成尽可能多样的对象操纵任务。因此,本研究的主要内容是:(1)头眼协同的场景与目标浏览;(2)头眼协同的目标选择;(3)头眼协同的对象操纵。大致研究步骤是:(1)确定针对操纵速度、准确度和使用负担的评估指标;(2)探索比较多种头眼协同交互方法;(3)确定一套高效易用的基于头眼协同的虚拟现实对象操纵方法,支持操纵对象位移、旋转和缩放;(4)根据评估指标对比目前国际一流方法(baseline)并作结果分析。

\section{论文结构}

本论文由五个章节构建而成。

\begin{itemize}
	\item {\bf 绪论}:本章节陈述了该课题来源以及其研究背景与意义,阐释了本次研究的目标,并对论文框架结构作出了介绍;
	\item {\bf 研究现状}:本章节梳理了基于手部(含手柄)的和基于眼动的对象操纵方法的国内外研究现状以及有代表性的工作;
	\item {\bf 方法设计与实现}:本章节详细介绍了本文提出的基于头眼协同的对象操纵方法,包括方法流程、场景浏览与目标选择、“四叶草”模式选择菜单、对象操纵以及信号处理;
	\item {\bf 实验设计}:本章节详细介绍了本次研究所涉及的两个先导实验和两个用户实验以及它们的设计意图和理念;
	\item {\bf 结论}:本章节概括了本次研究的结论和贡献,同时总结了本方法的不足之处及后续可开展的工作。
\end{itemize}