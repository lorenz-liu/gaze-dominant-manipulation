\chapter{简介}

\section{课题来源}

课题来源于北京航空航天大学虚拟现实技术与系统国家重点实验室。

\section{研究背景}

虚拟现实(Virtual Reality)技术,简称 VR,是一种利用计算机技术来模拟生活环境或创造虚拟现实的新型多媒体技术,是扩展现实(Extended Reality)技术的一个分支。目前主流的 VR 设备可利用头戴式显示器建立起一个完全虚拟的三维空间。使用者在这个虚拟的环境里进行交互操作时,计算机可以立即进行高度实时的、复杂的运算,将精确的三维影像传回,让使用者身处完全的沉浸式视觉环境中。该技术整合了计算机图形学、仿真模拟、人工智能以及并行计算等技术的最新发展成果,是一种融合多种先进技术的模拟系统。


\section{研究意义}

VR 已经在影视娱乐、教研教学、设计辅助等领域颇有建树,然而学界和工业界依旧留存着许多非常关键的问题亟待解决,拥有非常大的研究价值。

就目前而言,限制 VR 普及和发展的较为直接的阻碍,除开较高的市场售价,则来自于其仍旧较低的易用性,即虚拟环境中对物体的操控和交互的准确度依旧不容乐观,或是操作指令和交互动作过于复杂繁琐。这个缺陷直接降低了用户对 VR 技术的接受度和使用期望。因此,VR 中的对象操纵方法的优越性是提高其使用体验和普及度的基本问题之一。许多研究者已经进行了大量的研究,但仍有较大的提升空间。常规的对象操纵动作包括点击、按压、抓取、移动、释放等,其对应对象的直接具体表现主要为创建、销毁、位移、形变、旋转和缩放。对象操纵的速度、准确性、学习成本、使用压力和多样性将直接影响应用程序的效果,而在虚拟环境中实现高效且易用的对象操纵具有一定的挑战性。因此,本研究拟针对虚拟现实中对象操纵的关键问题进行研究,旨在提出相较于目前国际一流水准方法更加高效易用的基于头眼协同的虚拟现实对象操纵方法。
